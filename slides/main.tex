\documentclass[xcolor={dvipsnames},aspectratio=149]{beamer}

\definecolor{bgcolor}{rgb}{0.4,0.4,0.4}

% presentation parameters
\mode<presentation>
{
  \usetheme[block=fill]{metropolis}
  \usecolortheme{dove}
  \useoutertheme{infolines}
  \setbeamertemplate{headline}{}
  \setbeamertemplate{navigation symbols}{}
  \setbeamercolor{frametitle}{bg=bgcolor, fg=white}
  \setbeamercolor{author in head/foot}{fg=white,bg=bgcolor}
  \setbeamercolor{date in head/foot}{fg=white,bg=bgcolor}
  \setbeamercolor{title in head/foot}{fg=white,bg=bgcolor}
  \setbeamerfont{footnote}{size=\scriptsize}
}

%% % fontawesome fix
\defaultfontfeatures{Path = /usr/local/texlive/2023/texmf-dist/fonts/opentype/public/fontawesome/}

% packages
\usepackage{booktabs}
\usepackage{diagbox}
\usepackage{fontawesome}
\usepackage{minted}
\usepackage{multirow}
\usepackage{pifont}
\usepackage{xspace}
\usepackage{mathtools}
\usepackage{tikz}
\usepackage{caption}
\usepackage[normalem]{ulem}
\captionsetup{labelformat=empty}

% graphicx setup
\graphicspath{{figs/}}
\DeclareGraphicsExtensions{.pdf}


% boilerplate definitions
\def\eg{\emph{e.g.,}\xspace}
\def\etc{\emph{etc.}\xspace}
\def\ie{\emph{i.e.,}\xspace}
\def\etal{\emph{et al.}\xspace}

% styling
\def\bl{$\bullet$\xspace}		% bullet
\def\sq{{\tiny $\blacksquare$}\xspace}	% square
\def\tr{$\blacktriangleright$\xspace}	% triangle
\def\chk{\faCheck\xspace}		% check mark
\def\x{\faTimes\xspace}			% x mark
\def\st{\ding{74}\xspace}		% star mark
\def\rarr{\ding{220}\xspace}		% arrow (right)
\def\therefore{\ding{229}\xspace}
\newcommand{\cmark}{\ding{51}}%
\newcommand*\circled[1]{\tikz[baseline=(char.base)]{
  \node[shape=circle,draw,inner sep=1pt] (char) {#1};}}

% title
\title[Benchmarking]
{
  Benchmarking
}

% subtitle
\subtitle
{
  Goals and Strategies
}

% author
\author[\href{mailto:di_jin@brown.edu}{Di Jin}]
{
  \href{https://sleepymug.me/}{\textbf{Di Jin}}
}

% affiliation
% \institute[\href{https://cs.brown.edu}{Brown University}]
% {
% 	\href{https://gitlab.com/brown-ssl/}{Secure Systems Laboratory (SSL)} \\
% 	\href{https://cs.brown.edu}{Department of Computer Science} \\
% 	\href{https://www.brown.edu}{Brown University}
% }

% date
\date[CSCI2952R]			% (optional)
{
	% July 11, 2022
}

% aux
\subject{Talks}				% this is only inserted into the
					% PDF information catalog. Can be
					% left out

% logo
\pgfdeclareimage[height=1cm]{brown-logo}{figs/coat.png}
\logo{\pgfuseimage{brown-logo}}

% set the logo to nothing
\newcommand{\nologo}{\setbeamertemplate{logo}{}}

% non-numeric footnotes
%\renewcommand{\thefootnote}{\fnsymbol{footnote}}


\begin{document}

\begin{frame}
\titlepage
\end{frame}

\section{Basics}
\begin{frame}
  \frametitle{Realism}
  Use the most realistic benchmark you can get. It can (and should) also impact your system design.
  \begin{itemize}
  \item Domain-specific advantages in real-world examples.
    \begin{itemize}
    \item Good unit tests
    \item Tracing infrastructure
    \end{itemize}
  \item Accurate representation of the problems
    \begin{itemize}
    \item Scale
    \item Access patterns
    \item Bottlenecks (\eg IO, network)
    \end{itemize}
  \item Acquisition
    \begin{itemize}
    \item Research papers (selected benchmarks)
    \item Popular open-source projects
    \item Research papers (artifact)
    \end{itemize}
  \end{itemize}
\end{frame}

\begin{frame}
  \frametitle{Be Complete and Be Frank}
  Ultimate goal: allowing readers to extrapolate what will happen with their workload.
  \begin{itemize}
  \item Show good things
    \begin{itemize}
    \item Most ideal scenario/usecase
    \item Best workload
    \end{itemize}
  \item Show the bad things
    \begin{itemize}
    \item When does the scalability stop
    \item What workload will be negatively impacted
    \end{itemize}
  \item Show the performance breakdown
    \begin{itemize}
    \item Different component's overhead scales differently
    \item Future research can aim to improve specific part
    \item Readers can confirm their understanding
    \end{itemize}
  \end{itemize}
\end{frame}

\begin{frame}
  \frametitle{What to Benchmark?}
  Extraordinary claims require extraordinary evidence.
  \begin{itemize}
  \item Something extremely surprising needs a lot of evidence
    \begin{itemize}
    \item \eg my system uses a single core but it's faster than a multi-threaded solution
    \item Is there a breakdown?
    \item How does it work on different types of workload?
    \item What's the CPU utilization?
    \item What are the potential limits?
    \item $\ldots$
    \end{itemize}
  \item Something apparently true may require no benchmarking
    \begin{itemize}
    \item \eg my system adds metadata to \texttt{pip} packages for supply-chain security,
      it incurs no runtime memory overhead
    \end{itemize}
  \end{itemize}
\end{frame}

\section{Infrastructure}
\begin{frame}
  \frametitle{General Setup}
  \begin{itemize}
  \item Consider use managed server instances (\eg CloudLab \url{https://www.cloudlab.us/}, AWS)
  \item Daemonize the benchmarking process (\eg \texttt{screen}, \texttt{tmux}, \texttt{nohup})
  \item Repeatable
    \begin{itemize}
    \item Setup correct environment
    \item Avoid stale states
    \item \textit{Seriously} consider things like Docker
    \end{itemize}
  \end{itemize}
\end{frame}

\begin{frame}
  \frametitle{Anticipation for Problems}
  The moment all benchmarks run without any issue is probably the last time you run them.
  Anticipate problems, and be prepared to solve them.
  \begin{itemize}
  \item Granularity: don't put all tests into one scripts
  \item Fast iteration
    \begin{itemize}
    \item Have small scale benchmarks to help catch problems early (run in seconds)
    \item Have big scale benchmarks for main result (run this remotely, maybe hours or days)
    \end{itemize}
  \end{itemize}
\end{frame}

\begin{frame}
  \frametitle{Validity}
  ``Watch it fail first''
  \begin{itemize}
  \item Be very very paranoid
  \item Verify whether things work as expected
    \begin{itemize}
    \item Rough performance numbers
    \item Deliberately triggered failures
    \item Workload with predictably different outcomes
    \item Manual inspections (\eg debuggers, logging)
    \end{itemize}
  \end{itemize}
\end{frame}

\begin{frame}
  \frametitle{Automation}
  \begin{itemize}
  \item Push-button effort
    \begin{itemize}
    \item The benchmarks will be run \textbf{a lot}.
    \item If you are lucky, they will be run by other people as well.
    \end{itemize}
  \item Separate data generation and presentation
    \begin{itemize}
    \item Allows analysis and inspection without re-running the experiment
    \item Allows experimentation with data presentation when writing the paper
    \end{itemize}
  \item Only add abstractions as needed. Avoid over-design.
  \end{itemize}
\end{frame}

\section{Some Points on Methodology}
\begin{frame}
  \frametitle{No Interference}
  Do not run other things on the test environment
  \begin{itemize}
  \item No personal laptop (GUI, notifications, daemons)
  \item No running web server
  \item No resource sharing (even when VM is appropriate)
  \end{itemize}
\end{frame}

\begin{frame}
  \frametitle{Statistical Reporting}
  Report average of \textbf{repeated} experiments, and report \textbf{standard deviation} or
  \textbf{confidence interval}
  \begin{itemize}
  \item Not acceptable to say 5 second improvement with 20 second variation
  \item \sout{Pick a good one} (You would go to academic jail for this)
  \item \sout{Repeat 64 times to bring P-value down}
  \item Figure out why it's varying so much and fix it
    \begin{itemize}
    \item Interference
    \item Networking in non-networking benchmarks
    \item CPU auto-scaling/temperature throttling
    \end{itemize}
  \end{itemize}
\end{frame}

\begin{frame}
  \frametitle{Get Some Results First}
  Get 3--5 benchmarks end-to-end first
  \begin{itemize}
  \item Avoid fundamental problems
    \begin{itemize}
    \item Do not go three months into a project then realize that Python's GIL is a problem
    \end{itemize}
  \item Prioritize important things: \textit{``If a tree falls in a forest and no one is around to hear it, does it make a sound?''}

    \begin{itemize}
    \item Solve problems impacting the benchmarks
    \item If you believe something is important, find corresponding benchmark first
    \end{itemize}
  \end{itemize}
\end{frame}

\begin{frame}
  \frametitle{Other Things}
  \begin{itemize}
  \item Add additional tracing for to breakdown overhead
  \item Beware of the bottleneck of the system
    \begin{itemize}
    \item \eg don't hide CPU overhead under network bottleneck
    \end{itemize}
  \item Differentiate latency and throughput
  \end{itemize}
\end{frame}

\begin{frame}
  \frametitle{Further Readings}
    \textit{Systems Benchmarking Crimes} by Gernot Heiser \url{https://gernot-heiser.org/benchmarking-crimes.html}
\end{frame}
\end{document}
